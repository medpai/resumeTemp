\documentclass[letterpaper,11pt]{article}
\usepackage[utf8]{inputenc}
\usepackage[T1]{fontenc}
\usepackage[french]{babel}
\usepackage[hidelinks]{hyperref}
\usepackage[margin=1in]{geometry}
\setlength{\parindent}{0pt}
\setlength{\parskip}{6pt}

\begin{document}

Elmahdi Harchi\\
Montréal, QC\\
Tél.: +1 343-254-6991\\
Courriel: \href{mailto:mehdi.hr111@gmail.com}{mehdi.hr111@gmail.com}

15 août 2025\\

Commission de la construction du Québec (CCQ)\\
Direction – Exploitation\\
Centre de soutien informatique\\
Montréal, QC

Objet: Candidature — Technicien(ne) en informatique (Affichage 3403)

Madame, Monsieur,

Bilingue français/anglais et orienté centre de services, je vous soumets ma candidature au poste de Technicien(ne) en informatique (affichage 3403). Disponible immédiatement pour un mandat temporaire de 6 mois, je possède une expérience pratique en gestion des incidents et des demandes via un outil de billetterie, en soutien applicatif et en préparation/déploiement de postes Windows, en phase avec le rôle au Centre de soutien informatique de la CCQ.

Mes compétences clés en lien direct avec vos attentes:
- Centre de services: réception, enregistrement et suivi des demandes (en personne, téléphone, courriel, portail); communication claire et avis aux utilisateurs; escalades hiérarchiques et fonctionnelles au besoin.
- Postes de travail: installation, configuration et mise à jour; déploiement d'OS Windows 7/8/10/11; correctifs et vérifications post-déploiement.
- Accès et environnements: Active Directory (réinitialisation de mots de passe, permissions de base), VPN, RDP; Microsoft 365/Office 365 et notions d'Exchange; notions d'environnements serveurs Windows (2008/2012/2016).
- Qualité et amélioration continue: documentation de procédures et mise à jour de la base de connaissances; gestion d'inventaire et suivi des licences (notions); contribution aux suggestions d'amélioration et à la collaboration inter-équipes.

J'accorde une importance particulière à la courtoisie et à la bienséance dans mes interactions, à la rigueur dans le traitement des tickets et à la traçabilité des interventions. Ma pratique inclut l'appui aux utilisateurs pour l'usage de base des outils bureautiques (Outlook, Word, Excel, Teams) et la rédaction de guides simples et réutilisables. Ces habitudes me permettent de soutenir efficacement les opérations tout en améliorant l'expérience client interne.

Rejoindre la CCQ représente pour moi l'opportunité de contribuer à une mission d'intérêt public, au sein d'une équipe reconnue et d'un environnement structuré. Je serais heureux d'échanger pour vous présenter plus en détail ma démarche et mes réalisations.

Je vous remercie de l'attention portée à ma candidature et me tiens à votre disposition pour une entrevue.\
Veuillez agréer, Madame, Monsieur, l'expression de mes salutations distinguées.

\vspace{10mm}
Elmahdi Harchi

\end{document}
